%%%%%%%%%%%%%%%%%%%%%%%%%%%%%%%%%%%%%%%%%
% Medium Length Professional CV
% LaTeX Template
% Version 2.0 (8/5/13)
%
% This template has been downloaded from:
% http://www.LaTeXTemplates.com
%
% Original author:
% Trey Hunner (http://www.treyhunner.com/)
%
% Important note:
% This template requires the resume.cls file to be in the same directory as the
% .tex file. The resume.cls file provides the resume style used for structuring the
% document.
%
%%%%%%%%%%%%%%%%%%%%%%%%%%%%%%%%%%%%%%%%%

%----------------------------------------------------------------------------------------
%	PACKAGES AND OTHER DOCUMENT CONFIGURATIONS
%----------------------------------------------------------------------------------------

\documentclass{resume} % Use the custom resume.cls style

\usepackage[left=0.75in,top=0.6in,right=0.75in,bottom=0.6in]{geometry} % Document margins

\name{Jordan Chapman} % Your name
\address{Avaialable Upon Request \\ North Carolina} % Your address
\address{(555)~$\cdot$~555~$\cdot$~5555 \\ jchapman134569@gmail.com} % Your phone number and email

\begin{document}

%----------------------------------------------------------------------------------------
%	EDUCATION SECTION
%----------------------------------------------------------------------------------------

\begin{rSection}{Education}

{\bf Western Carolina University} \hfill {\em 2011 -2013} \\ 
{\bf Double Major Bachelor of Science} \hfill {\em Cullowhee, NC} \\
Computer Science \& Mathematics \hfill {\em GPA: 3.463} \smallskip \\
President of WCU's Student Affiliate Chapter of the ACM\\
President of WCU's Student Chapter of the Mathematic Club of America
\end{rSection}


%----------------------------------------------------------------------------------------
%    TECHNICAL STRENGTHS SECTION
%----------------------------------------------------------------------------------------

\begin{rSection}{Technical Skills}

\begin{tabular}{ @{} >{\bfseries}l @{\hspace{6ex}} l }
Languages & Java, C, C++, Python, JavaScript, Bash, SQL, Perl, MIPS, \\
& Adobe Flex, HTML, CSS, XML, PHP  \\

Software Tools & Git, Ant, Make, Vim, Eclipse, Visual Studio, CMake, Subversion,\\ 
& CVS, Rational ClearCase, JUnit, Mercurial, GDB, Awk, Sed \\

Platforms & Linux, Windows, Android, Mac OS \\

Other & Agile Development (Scrum), Django, OpenCV, \LaTeX , Swing, \\
& Google Maps API, Inkscape \\
\end{tabular}
    
\end{rSection}

%----------------------------------------------------------------------------------------
%	WORK EXPERIENCE SECTION
%----------------------------------------------------------------------------------------

\begin{rSection}{Work Experience}

\begin{rSubsection}{Applied Research Associates}{May 2013 - Present}{Junior Software Engineer}{Raleigh, NC}
\item Developed widgets for Esri Flex Viewer which displayed standoff distances for chemicals and IEDs.
\item Widget functionality included displaying an optimal set of roadblocks to isolate the affected area.
\item Worked as part of a team maintaining and developing a large program which simulates many scenarios containing explosives.
\end{rSubsection}

%------------------------------------------------
\begin{rSubsection}{Thomson Reuters}{May 2012 - August 2012}{Software Development Intern}{Durham, NC}
\item Helped develop and maintain a web-based patient tracking system.
\item Worked directly with a team in an Agile development environment.
\item Worked focused on how to deliver updated content to customers with unknown versions of content.
\end{rSubsection}

%------------------------------------------------



\begin{rSubsection}{Western Carolina University}{August 2011 - May 2012}{Android Developer}{Cullowhee, NC}
\item Worked with the Study of Developed Shorelines under NOAA.
\item Developed an application to visualize data about the impact of hurricanes.
\item Gained a solid understanding of the Android operation system as well as concerns when developing for mobile devices.
\item Reinforced knowledge of databases, design patterns, and customer interaction.
\item An article about the application was published in Scientific American. 
\end{rSubsection}
\end{rSection}


%----------------------------------------------------------------------------------------
%    Projects SECTION
%----------------------------------------------------------------------------------------

\begin{rSection}{Selected Projects}

{\bf Hazard Detection and Avoidance Using OpenCV}
\begin{list}{$\cdot$}{\leftmargin=0em} % \cdot used for bullets, no indentation
   \itemsep -0.5em \vspace{-0.5em} % Compress items in list together for aesthetics
    \item Dynamically identified and tracked hazards and goals in a video game using the OpenCV library in order to create an AI able to solve each level.
    \item Created an AI which interpreted the data tracked by OpenCV and used it to complete simple levels.
\end{list}

%{\bf Creating a New Inter-Process Communication Mechanism for Linux}
%\begin{list}{$\cdot$}{\leftmargin=0em} % \cdot used for bullets, no indentation
%   \itemsep -0.5em \vspace{-0.5em} % Compress items in list together for aesthetics
%    \item Created a queue in each process that could be written to or read from.
%    \item Worked with Linux 3.0.6, code was all in kernel space.
%    \item Gained a deeper understanding of Linux, how it works, and how to modify it.
%\end{list}
\end{rSection}

%----------------------------------------------------------------------------------------

%----------------------------------------------------------------------------------------
%	EXAMPLE SECTION
%----------------------------------------------------------------------------------------

%\begin{rSection}{Section Name}

%Section content\ldots

%\end{rSection}

%----------------------------------------------------------------------------------------

\end{document}
